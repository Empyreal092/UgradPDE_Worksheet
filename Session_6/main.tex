\documentclass[11pt,letterpaper]{article}
\usepackage[utf8]{inputenc}
\usepackage[left=1in,right=1in,top=1in,bottom=1in]{geometry}
\usepackage{amsfonts,amsmath}
\usepackage{graphicx,float}
\usepackage{csquotes}
% -----------------------------------
\usepackage{hyperref}
\hypersetup{%
  colorlinks=true,
  linkcolor=blue,
  citecolor=blue,
  urlcolor=blue,
  linkbordercolor={0 0 1}
}
% -----------------------------------
\usepackage[authordate,backend=biber]{biblatex-chicago}
\addbibresource{citation.bib}
% -----------------------------------
\usepackage{fancyhdr}
\newcommand\course{MATH-UA.0263\\Partial Differential Equations}
\newcommand\hwnumber{6}                  % <-- homework number
\newcommand\NetIDa{Ryan Sh\`iji\'e D\`u} 
\newcommand\NetIDb{March 10th, 2023}
\pagestyle{fancyplain}
\headheight 35pt
\lhead{\NetIDa\\\NetIDb}
\chead{\textbf{\Large Worksheet \hwnumber}}
\rhead{\course}
\lfoot{}
\cfoot{}
\rfoot{\small\thepage}
\headsep 1.5em
% -----------------------------------
\usepackage{titlesec}
\renewcommand\thesubsection{(\arabic{section}.\alph{subsection})}
\titleformat{\subsection}[runin]
        {\normalfont\bfseries}
        {\thesubsection}% the label and number
        {0.5em}% space between label/number and subsection title
        {}% formatting commands applied just to subsection title
        []% punctuation or other commands following subsection title
% -----------------------------------
\setlength{\parindent}{0.0in}
\setlength{\parskip}{0.1in}
% -----------------------------------
\newcommand{\de}{\mathrm{d}}
\newcommand{\DD}{\mathrm{D}}
\newcommand{\pe}{\partial}
\newcommand{\mcal}{\mathcal}
%\newcommand{\pdx}{\left|\frac{\partial}{\partial_x}\right|}

\newcommand{\dsp}{\displaystyle}

\newcommand{\norm}[1]{\left\Vert #1 \right\Vert}
%\newcommand{\mean}[1]{\left\langle #1 \right\rangle}
\newcommand{\mean}[1]{\overline{#1}}
\newcommand{\inner}[2]{\left\langle #1,#2\right\rangle}

\newcommand{\ve}[1]{\boldsymbol{#1}}

\newcommand{\thus}{\Rightarrow \quad }
\newcommand{\fff}{\iff\quad}
\newcommand{\qdt}[1]{\quad \mbox{#1} \quad}

\renewcommand{\Re}{\mathrm{Re}}
\renewcommand{\Im}{\mathrm{Im}}
\newcommand{\E}{\mathbb{E}}
\newcommand{\lap} {\nabla^2}
\renewcommand{\div}{\nabla\cdot}

\newcommand{\csch}{\text{csch}}
\newcommand{\sech}{\text{sech}}


\newcommand{\hot}{\text{h.o.t.}}

\newcommand{\ssp}{\left.\qquad\right.}

\newcommand{\var}{\text{var}}
\newcommand{\cov}{\text{cov}}


\begin{document}

\section{Heat flux and the Robin boundary condition}
\subsection{}
Write the heat equation as a conservation law. What is the heat flux function? 

Relating the flux to its gradient is called the Fourier's law or the Fick's law.

\subsection{}
Imagine we have a metal rod where one end of it is submerged in cold water. The water has temperature $T_L$ and is moving. The heat flux due to convection is
\begin{align}
    q = h(T-T_L)
\end{align}
where $h$ is called the convection coefficient.

This heat flux must be equal to the heat flux at the end of rod which follows the Fourier's law. From this obtain that $\theta = T-T_L$ must satisfy the homogeneous robin boundary condition.

% \section{Application of the fundamental solution to solve Dirichlet problems}
% \subsection{}
% Error function

% \subsection{}
% Combine them

% \subsection{}
% [From \cite{Olver_14}, Problem 8.1.1]

\section{Alternative statements of the method of images}
[From \cite{ShearerLevy_15}, Problem 5.3] We have $\Phi$ the fundamental solution of the heat equation.

\subsection{}
Let $g:[0,\infty)\to\mathbb{R}$ be a bounded integrable function. Prove directly that
\begin{align}
    u(x,t)=\int_0^\infty (\Phi(x-y,t)-\Phi(x+y,t))g(y)\;\de y
\end{align}
is an odd function of $x\in\mathbb{R}$ for each $t>0$.

\subsection{}
Let $h:\mathbb{R}\to\mathbb{R}$ be an odd bounded integrable function. Prove that
\begin{align}
    u(x,t)=\int^\infty_{-\infty} \Phi(x-y,t)h(y)\;\de y
\end{align}
is an odd function of $x\in\mathbb{R}$ for each $t > 0$. That is, the symmetry in the initial data is carried through to the same symmetry in the solution.

\section{Maximum and variance of the fundamental solution}
[Adapted from \cite{Olver_14}, Problem 8.1.6]

\subsection{}
What is the maximum value of the fundamental solution at time $t$?

\subsection{}
Calculate the ``variance'' of the fundamental solution:
\begin{align}
    \var(u(t)) = \int_\mathbb{R} x^2 u(t,x)\;\de x.
\end{align}
One could do this directly. Alternatively, calculate how the variance change in time:
\begin{align}
    \frac{\de}{\de t}\var(u(t)) = \frac{\de}{\de t}\int_\mathbb{R} x^2 u(t,x)\;\de x.
\end{align}

\subsection{}
Can you justify the claim that its width is proportional to $\sqrt{t}$?

\section{Advection-diffusion equation}
[From \cite{ShearerLevy_15}, Problem 5.10] Devise a change of variable corresponding to a moving frame of reference to solve the initial value problem for the advection-diffusion equation with constant speed $c$
\begin{align}
    &u_t+cu_x = ku_{xx},\quad -\infty<x<\infty, t>0,\\
    &u(x,0) = g(x),\quad -\infty<x<\infty
\end{align}

\section{Heat equation with damping}
[From \cite{ShearerLevy_15}, Problem 5.9] Consider the initial value problem
\begin{align}
    &u_t+du = ku_{xx},\quad -\infty<x<\infty, t>0,\\
    &u(x,0) = g(x),\quad -\infty<x<\infty
\end{align}
with constant $d$, and given integrable function $g$.

\subsection{}
Use the change of variable $u(x, t) = e^{-dt}v(x,t)$ to find $u$ using the
fundamental solution.

\subsection{}
What is the effect of the constant $d$?

\subsection{}
Suppose $d = d(t)$ is a given continuous function. What would be a suitable
change of variable to solve the problem?


\vfill
\printbibliography


\end{document}