\documentclass[11pt,letterpaper]{article}
\usepackage[utf8]{inputenc}
\usepackage[left=1in,right=1in,top=1in,bottom=1in]{geometry}
\usepackage{amsfonts,amsmath}
\usepackage{graphicx,float}
\usepackage{csquotes}
\usepackage{esint}
% -----------------------------------
\usepackage{hyperref}
\hypersetup{%
  colorlinks=true,
  linkcolor=blue,
  citecolor=blue,
  urlcolor=blue,
  linkbordercolor={0 0 1}
}
% -----------------------------------
\usepackage[style=authoryear-icomp,backend=biber]{biblatex}
\addbibresource{citation.bib}
% -----------------------------------
\usepackage{fancyhdr}
\newcommand\course{MATH-UA.0263\\Partial Differential Equations}
\newcommand\hwnumber{10}                  % <-- homework number
\newcommand\NetIDa{Ryan Sh\`iji\'e D\`u} 
\newcommand\NetIDb{April 14th, 2023}
\pagestyle{fancyplain}
\headheight 35pt
\lhead{\NetIDa\\\NetIDb}
\chead{\textbf{\Large Worksheet \hwnumber}}
\rhead{\course}
\lfoot{}
\cfoot{}
\rfoot{\small\thepage}
\headsep 1.5em
% -----------------------------------
\usepackage{titlesec}
\renewcommand\thesubsection{(\arabic{section}.\alph{subsection})}
\titleformat{\subsection}[runin]
        {\normalfont\bfseries}
        {\thesubsection}% the label and number
        {0.5em}% space between label/number and subsection title
        {}% formatting commands applied just to subsection title
        []% punctuation or other commands following subsection title
% -----------------------------------
\setlength{\parindent}{0.0in}
\setlength{\parskip}{0.1in}
% -----------------------------------
\input{../command.tex}
\begin{document}

\section{Poisson’s equation with Robin boundary conditions}
[From \cite{ShearerLevy_15}, Problem 8.3]
Consider Poisson’s equation on a bounded open set $U \in\mathbb{R}^n$ with Robin boundary conditions
\begin{align}
    &\nabla^2 u = f(\ve x),\quad\ve x\in U,\\
    &\qdt{with} \frac{\pe u}{\pe\ve n}+\alpha u(\ve x) = g(\ve x),\quad\ve x\in\pe U.
\end{align}

\subsection{}
Prove that if $\alpha > 0$, then the energy method can be used to show
uniqueness of solutions $u\in C^2(U)\cap C(\overline{U})$

\subsection{}
For $\alpha = 0$, show that solutions are unique up to a constant.

\subsection{}
Design an example to show that uniqueness can fail if $\alpha < 0$. (Hint: Choose $n = 1$.)

\section{Conservation laws of KdV}
[From \cite{ShearerLevy_15}, Problem 12.2] Given the KdV equation
\begin{align}
    u_t+uu_x+\gamma u_{xxx} = 0
\end{align}
with the condition that the solution decays sufficiently rapidly as $|x|\to\infty$.

\subsection{}
Show that the momentum and kinetic energy (when $u$ interpreted as a velocity) are conserved:
\begin{align}
    \int_\mathbb{R} u\;\de x; \qquad \int_\mathbb{R} u^2\;\de x.
\end{align}

\subsection{}
Find $\eta$ (depending on $\gamma$) so that the integral
\begin{align}
    \int_\mathbb{R} \frac{1}{2}(u_x^2-\eta u^3)\;\de x
\end{align}
is conversed. 

\section{Separation of variables for the Helmholtz equation}
[From \cite{Olver_14}, Problem 4.3.18] Use separation of variables to solve the Helmholtz boundary value problem on the unit square:
\begin{align}
    &\nabla^2 u = k^2u\\
    &\qdt{with} u(x,0)=0, u(x,1)=f(x), u(0, y) = 0, u(1, y) = 0
\end{align}

\section{Energy minimization}
[From Fall 2015 of Applied Differential Equations qualifying exam at UCLA, Problem 7\footnote{\url{https://ww3.math.ucla.edu/wp-content/uploads/2021/09/ade-15F.pdf}}] 
Consider $K$ be the set of functions $u: [0, 2] \to \mathbb{R}$ of the form 
\begin{align}
    u(x) = \begin{cases}
        v(x)\qdt{for} x\in[0,1)\\
        w(x)\qdt{for} x\in(1,2]
    \end{cases}
\end{align} 
and $v\in C^2[0,1)$, $w\in C^2(1,2]$, with the property that $v(0) = w(2) = 0$ and $[u] := w(1) - v(1) = a$. You should assume that the value of the constant $a$ is known. Define
\begin{align}
    E(u) = \frac{1}{2}\int_0^1 (u_x)^2\;\de x+\int^2_1 (u_x)^2\;\de x+\frac{w(1)+v(1)}{2}b
\end{align}
where $b$ is another constant, whose value you can assume is known. Show that there exists $h(x)$ that minimizes $E$ over all functions $u\in K$, and solve for $h(x)$.

\section{Green's identity and Green's function}
\subsection{}
We take a 3D scalar field $\psi$ and a 3D vector field $\ve\Gamma$ with sufficient smoothness defined on some region $U \subset \mathbb{R}^3$. Show the identity
\begin{align}
    \iiint_U \left( \psi \, \nabla \cdot \ve{\Gamma} + \ve{\Gamma} \cdot \nabla \psi\right)\, dV  = \oiint_{\partial U} \psi \left( \ve{\Gamma} \cdot \ve{n} \right)\, dS=\oiint_{\partial U}\psi\ve{\Gamma}\cdot d\ve{S}.\label{eq:green_1st_gen}
\end{align}

\subsection{}
Use \eqref{eq:green_1st_gen} to show the Green's first identity. Take 3D scalar fields $\psi$ and $\varphi$ both with sufficient smoothness:
\begin{align}
    \iiint_U \left( \psi \, \nabla^2 \varphi + \nabla \psi \cdot \nabla \varphi \right)\, dV  = \oiint_{\partial U} \psi \left( \nabla \varphi \cdot \ve{n} \right)\, dS=\oiint_{\partial U}\psi\,\nabla\varphi\cdot d\ve{S}.
\end{align}

\subsection{}
Show the Green's second identity:
\begin{align}
    \iiint_U \left( \psi \, \nabla^2 \varphi - \varphi \, \nabla^2 \psi\right)\, dV = \oiint_{\partial U} \left( \psi \nabla \varphi - \varphi \nabla \psi\right)\cdot d\ve{S}.
\end{align}
This shows that the Laplacian is a self-adjoint operator for functions vanishing on the boundary so that the right hand side of the above identity is zero.

\subsection{}
Suppose we have the Green's function of the Poisson equation on the bounded domain $\Omega$. That is, we have $G(\ve x;\ve y)$ s.t.
\begin{align}
    &\nabla_{\ve x}^2 G = \delta(\ve x-\ve y)\qdt{with BC} \left.G\right|_{\pe\Omega} = 0.
\end{align}
Use Green's second identity to obtain the solution to the Poisson equation
\begin{align}
    &\nabla_{\ve x}^2 u = f\qdt{with BC} \left.u\right|_{\pe\Omega} = g
\end{align}
from the Green's function.

Think about how would you get such a Green's function for general domains. 

\vfill
\printbibliography

\end{document}