\documentclass[11pt,letterpaper]{article}
\usepackage[utf8]{inputenc}
\usepackage[left=1in,right=1in,top=1in,bottom=1in]{geometry}
\usepackage{amsfonts,amsmath}
\usepackage{graphicx,float}
\usepackage{csquotes}
\usepackage{esint}
% -----------------------------------
\usepackage{hyperref}
\hypersetup{%
  colorlinks=true,
  linkcolor=blue,
  citecolor=blue,
  urlcolor=blue,
  linkbordercolor={0 0 1}
}
% -----------------------------------
\usepackage[style=authoryear-icomp,backend=biber]{biblatex}
\addbibresource{citation.bib}
% -----------------------------------
\usepackage{fancyhdr}
\newcommand\course{MATH-UA.0263\\Partial Differential Equations}
\newcommand\hwnumber{12}                  % <-- homework number
\newcommand\NetIDa{Ryan Sh\`iji\'e D\`u} 
\newcommand\NetIDb{April 28th, 2023}
\pagestyle{fancyplain}
\headheight 35pt
\lhead{\NetIDa\\\NetIDb}
\chead{\textbf{\Large Worksheet \hwnumber}}
\rhead{\course}
\lfoot{}
\cfoot{}
\rfoot{\small\thepage}
\headsep 1.5em
% -----------------------------------
\usepackage{titlesec}
\renewcommand\thesubsection{(\arabic{section}.\alph{subsection})}
\titleformat{\subsection}[runin]
        {\normalfont\bfseries}
        {\thesubsection}% the label and number
        {0.5em}% space between label/number and subsection title
        {}% formatting commands applied just to subsection title
        []% punctuation or other commands following subsection title
% -----------------------------------
\setlength{\parindent}{0.0in}
\setlength{\parskip}{0.1in}
% -----------------------------------
\newcommand{\de}{\mathrm{d}}
\newcommand{\DD}{\mathrm{D}}
\newcommand{\pe}{\partial}
\newcommand{\mcal}{\mathcal}
%\newcommand{\pdx}{\left|\frac{\partial}{\partial_x}\right|}

\newcommand{\dsp}{\displaystyle}

\newcommand{\norm}[1]{\left\Vert #1 \right\Vert}
%\newcommand{\mean}[1]{\left\langle #1 \right\rangle}
\newcommand{\mean}[1]{\overline{#1}}
\newcommand{\inner}[2]{\left\langle #1,#2\right\rangle}

\newcommand{\ve}[1]{\boldsymbol{#1}}

\newcommand{\thus}{\Rightarrow \quad }
\newcommand{\fff}{\iff\quad}
\newcommand{\qdt}[1]{\quad \mbox{#1} \quad}

\renewcommand{\Re}{\mathrm{Re}}
\renewcommand{\Im}{\mathrm{Im}}
\newcommand{\E}{\mathbb{E}}
\newcommand{\lap} {\nabla^2}
\renewcommand{\div}{\nabla\cdot}

\newcommand{\csch}{\text{csch}}
\newcommand{\sech}{\text{sech}}


\newcommand{\hot}{\text{h.o.t.}}

\newcommand{\ssp}{\left.\qquad\right.}

\newcommand{\var}{\text{var}}
\newcommand{\cov}{\text{cov}}


\begin{document}

\section{Green's identity and Green's function}
\subsection{}
We take a 3D scalar field $\psi$ and a 3D vector field $\ve\Gamma$ with sufficient smoothness defined on some region $U \subset \mathbb{R}^3$. Show the identity
\begin{align}
    \iiint_U \left( \psi \, \nabla \cdot \ve{\Gamma} + \ve{\Gamma} \cdot \nabla \psi\right)\, dV  = \oiint_{\partial U} \psi \left( \ve{\Gamma} \cdot \ve{n} \right)\, dS=\oiint_{\partial U}\psi\ve{\Gamma}\cdot d\ve{S}.\label{eq:green_1st_gen}
\end{align}

\subsection{}
Use \eqref{eq:green_1st_gen} to show the Green's first identity. Take 3D scalar fields $\psi$ and $\varphi$ both with sufficient smoothness:
\begin{align}
    \iiint_U \left( \psi \, \nabla^2 \varphi + \nabla \psi \cdot \nabla \varphi \right)\, dV  = \oiint_{\partial U} \psi \left( \nabla \varphi \cdot \ve{n} \right)\, dS=\oiint_{\partial U}\psi\,\nabla\varphi\cdot d\ve{S}.
\end{align}

\subsection{}
Show the Green's second identity:
\begin{align}
    \iiint_U \left( \psi \, \nabla^2 \varphi - \varphi \, \nabla^2 \psi\right)\, dV = \oiint_{\partial U} \left( \psi \nabla \varphi - \varphi \nabla \psi\right)\cdot d\ve{S}.
\end{align}
This shows that the Laplacian is a self-adjoint operator for functions vanishing on the boundary so that the right hand side of the above identity is zero.

\subsection{}
Suppose we have the Green's function of the Poisson equation on the bounded domain $\Omega$. That is, we have $G(\ve x;\ve y)$ s.t.
\begin{align}
    &\nabla_{\ve x}^2 G = \delta(\ve x-\ve y)\qdt{with BC} \left.G\right|_{\pe\Omega} = 0.
\end{align}
Use Green's second identity to obtain the solution to the Poisson equation
\begin{align}
    &\nabla_{\ve x}^2 u = f\qdt{with BC} \left.u\right|_{\pe\Omega} = g
\end{align}
from the Green's function.

Think about how would you get such a Green's function for general domains. 

\section{Energy minimization for Poisson and Laplace}
\subsection{} Suppose that we are interested in solving the Poisson's equation:
\begin{align}
    -\nabla^2 u = f(x)
\end{align}
for continuous $f(x)$, and with Dirichlet boundary conditions over $\pe\Omega$ for $u$. Show that the solution of this equation minimizes, among $u\in C^2(\Omega)\cap C(\overline{\Omega})$, with $u|_{\pe\Omega} = g(x)$, the \emph{complementary energy}:
\begin{align}
    V[u] = \int_\Omega \left(\frac{1}{2}\norm{\nabla u}^2-f(x)u\right)\de V. 
\end{align}

Inversely, show that the minimizer of $V$ solves the Poisson's equation.

\subsection{}
Now we study Laplace with \emph{Neumann boundary condition}. Suppose that the boundary can be divided into $\pe\Omega_D$ on which $u = g(x)$ is known, and $\pe\Omega_N$ on
which $\pe u/\pe n = N(x)$ is known. Show that $u$ minimizes:
\begin{align}
    W[u] = \int_\Omega \frac{1}{2}\norm{\nabla u}^2\de V-\int_{\pe\Omega_N} Nu\;\de S. 
\end{align}
among $u\in C^2(\Omega)\cap C(\overline{\Omega})$, with $u|_{\pe\Omega_D} = g(x)$. Additionally, show the inverse.

\section{Euler–Tricomi equation}
The Euler–Tricomi equation is a PDE useful in the study of transonic flow:
\begin{align}
    u_{xx}+xu_{yy} = 0.
\end{align}
This equation changed its type over $x$. Categorize this equation.



\vfill
\printbibliography

\end{document}