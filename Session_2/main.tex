\documentclass[11pt,letterpaper]{article}
\usepackage[utf8]{inputenc}
\usepackage[left=1in,right=1in,top=1in,bottom=1in]{geometry}
\usepackage{amsfonts,amsmath}
\usepackage{graphicx,float}
% -----------------------------------
\usepackage{hyperref}
\hypersetup{%
  colorlinks=true,
  linkcolor=blue,
  citecolor=blue,
  urlcolor=blue,
  linkbordercolor={0 0 1}
}
% -----------------------------------
\usepackage[authordate,backend=biber]{biblatex-chicago}
\addbibresource{citation.bib}
% -----------------------------------
\usepackage{fancyhdr}
\newcommand\course{MATH-UA.0263\\Partial Differential Equations}
\newcommand\hwnumber{2}                  % <-- homework number
\newcommand\NetIDa{Ryan Sh\`iji\'e D\`u} 
\newcommand\NetIDb{February 10th, 2023}
\pagestyle{fancyplain}
\headheight 35pt
\lhead{\NetIDa\\\NetIDb}
\chead{\textbf{\Large Worksheet \hwnumber}}
\rhead{\course}
\lfoot{}
\cfoot{}
\rfoot{\small\thepage}
\headsep 1.5em
% -----------------------------------
\usepackage{titlesec}
\renewcommand\thesubsection{(\arabic{section}.\alph{subsection})}
\titleformat{\subsection}[runin]
        {\normalfont\bfseries}
        {\thesubsection}% the label and number
        {0.5em}% space between label/number and subsection title
        {}% formatting commands applied just to subsection title
        []% punctuation or other commands following subsection title
% -----------------------------------
\setlength{\parindent}{0.0in}
\setlength{\parskip}{0.1in}
% -----------------------------------
\newcommand{\de}{\mathrm{d}}
\newcommand{\DD}{\mathrm{D}}
\newcommand{\pe}{\partial}
\newcommand{\mcal}{\mathcal}
%\newcommand{\pdx}{\left|\frac{\partial}{\partial_x}\right|}

\newcommand{\dsp}{\displaystyle}

\newcommand{\norm}[1]{\left\Vert #1 \right\Vert}
%\newcommand{\mean}[1]{\left\langle #1 \right\rangle}
\newcommand{\mean}[1]{\overline{#1}}
\newcommand{\inner}[2]{\left\langle #1,#2\right\rangle}

\newcommand{\ve}[1]{\boldsymbol{#1}}

\newcommand{\thus}{\Rightarrow \quad }
\newcommand{\fff}{\iff\quad}
\newcommand{\qdt}[1]{\quad \mbox{#1} \quad}

\renewcommand{\Re}{\mathrm{Re}}
\renewcommand{\Im}{\mathrm{Im}}
\newcommand{\E}{\mathbb{E}}
\newcommand{\lap} {\nabla^2}
\renewcommand{\div}{\nabla\cdot}

\newcommand{\csch}{\text{csch}}
\newcommand{\sech}{\text{sech}}


\newcommand{\hot}{\text{h.o.t.}}

\newcommand{\ssp}{\left.\qquad\right.}

\newcommand{\var}{\text{var}}
\newcommand{\cov}{\text{cov}}


\begin{document}

\section{Domain of dependence}
[From \cite{ShearerLevy_15}, Exercise 4.6] Suppose $u$ satisfies the wave equation with $c=1$ for $x>0$. We want to find a solution of the PDE for $x>0, t>0$, that satisfy the initial conditions
\begin{align}
    & u(x,0) = \phi(x)\\
    & u_t(x,0) = \psi(x),
\end{align}
and the boundary condition
\begin{align}
    u_x(x,0) = 0.
\end{align}

\subsection{}
Solve for $u(x,t)$

\subsection{}
Assume $\supp\phi = \supp\psi=[1,2]$, where can you guarantee that $u=0$ for $x>0, t>0$. 

\section{Fourier series and the expansions of $\pi$}
\subsection{}
Show the Fourier series of periodic extension of $x$ on the interval $(-\pi,\pi)$ is
\begin{align}
    x = \sum_{n=1}^\infty \frac{2(-1)^{n+1}}{n}\sin(nx).
\end{align}

\subsection{}
Show this expansion of $\pi$:
\begin{align}
    \frac{\pi}{4} = 1-\frac{1}{3}+\frac{1}{5}-\frac{1}{7}+\dots
\end{align}

\subsection{}
Show the solution to the Basel problem is
\begin{align}
    1+\frac{1}{4}+\frac{1}{9}+\frac{1}{16} + \dots = \frac{\pi^2}{6}.
\end{align}

\section{Challenging wave equation problem}
[From Spring 2019 of Applied Differential Equations qualifying exam at UCLA, Problem 1\footnote{\url{https://ww3.math.ucla.edu/wp-content/uploads/2021/09/ade-19S.pdf}}]  Let $u(x,t)$ solve the initial value problem
\begin{align}
    \begin{cases}
        u_{tt}+u_{xt}-2u_{xx} = 0,\quad x\in\mathbb{R}, t>0,\\
        u(x,0) = g(x),\\
        u_t(x,0) = h(x).
    \end{cases}
\end{align}

\subsection{}
Derive a formula for $u$ in terms of $g$ and $h$, when $g$ and $h$ are $C^2$. 

Hint: Consider how to simplify the equation into something more obviously like the wave equation
by making a change of coordinate system: $(x,t)\to (\zeta,t)$ where $\zeta = x-vt$ for $v$ appropriately determined. 

\subsection{}
Next consider the boundary value problem 
\begin{align}
    \begin{cases}
        u_{tt}+u_{xt}-2u_{xx} = 0,\quad x\in[0,1], t>0,\\
        u(0,t) = u(1,t) = 0.
    \end{cases}
\end{align}
Show that a smooth solution $u$ to above problem must be zero if $u(x,0)=u_t(x,0)=0$. 

Hint: use an energy argument. Try the energy for the wave equation. 

\section{Equipartition of energy}
[From \cite{Evans_10}, Exercise 2.24] Let $u$ solve the initial-value problem for the wave equation in one dimension:
\begin{align}
    u_{tt}-u_{xx} = 0 &\qdt{in} \mathbb{R}\times(0,\infty)\\
    u=0, u_t=0 &\qdt{on} \mathbb{R}\times\{t=0\}.
\end{align}
Suppose $g,h$ have compact support. We define the kinetic energy
\begin{align}
    k(t):= \frac{1}{2}\int_{-\infty}^\infty u_t^2(x,t)\;\de x
\end{align}
and the potential energy
\begin{align}
    p(t):= \frac{1}{2}\int_{-\infty}^\infty u_x^2(x,t)\;\de x.
\end{align}
We know that the total energy $k(t)+p(t)$ is constant in $t$. Show that $k(t)=p(t)$ for all large enough times $t$. 

    
\vfill
\printbibliography


\end{document}