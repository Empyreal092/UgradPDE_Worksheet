\documentclass[11pt,letterpaper]{article}
\usepackage[utf8]{inputenc}
\usepackage[left=1in,right=1in,top=1in,bottom=1in]{geometry}
\usepackage{amsfonts,amsmath}
\usepackage{graphicx,float}
% -----------------------------------
\usepackage{hyperref}
\hypersetup{%
  colorlinks=true,
  linkcolor=blue,
  citecolor=blue,
  urlcolor=blue,
  linkbordercolor={0 0 1}
}
% -----------------------------------
\usepackage[authordate,backend=biber]{biblatex-chicago}
\addbibresource{citation.bib}
% -----------------------------------
\usepackage{fancyhdr}
\newcommand\course{MATH-UA.0263\\Partial Differential Equations}
\newcommand\hwnumber{1}                  % <-- homework number
\newcommand\NetIDa{Ryan Sh\`iji\'e D\`u} 
\newcommand\NetIDb{February 3rd, 2023}
\pagestyle{fancyplain}
\headheight 35pt
\lhead{\NetIDa\\\NetIDb}
\chead{\textbf{\Large Worksheet \hwnumber}}
\rhead{\course}
\lfoot{}
\cfoot{}
\rfoot{\small\thepage}
\headsep 1.5em
% -----------------------------------
\usepackage{titlesec}
\renewcommand\thesubsection{(\arabic{section}.\alph{subsection})}
\titleformat{\subsection}[runin]
        {\normalfont\bfseries}
        {\thesubsection}% the label and number
        {0.5em}% space between label/number and subsection title
        {}% formatting commands applied just to subsection title
        []% punctuation or other commands following subsection title
% -----------------------------------
\setlength{\parindent}{0.0in}
\setlength{\parskip}{0.1in}
% -----------------------------------
\newcommand{\de}{\mathrm{d}}
\newcommand{\DD}{\mathrm{D}}
\newcommand{\pe}{\partial}
\newcommand{\mcal}{\mathcal}
%\newcommand{\pdx}{\left|\frac{\partial}{\partial_x}\right|}

\newcommand{\dsp}{\displaystyle}

\newcommand{\norm}[1]{\left\Vert #1 \right\Vert}
%\newcommand{\mean}[1]{\left\langle #1 \right\rangle}
\newcommand{\mean}[1]{\overline{#1}}
\newcommand{\inner}[2]{\left\langle #1,#2\right\rangle}

\newcommand{\ve}[1]{\boldsymbol{#1}}

\newcommand{\thus}{\Rightarrow \quad }
\newcommand{\fff}{\iff\quad}
\newcommand{\qdt}[1]{\quad \mbox{#1} \quad}

\renewcommand{\Re}{\mathrm{Re}}
\renewcommand{\Im}{\mathrm{Im}}
\newcommand{\E}{\mathbb{E}}
\newcommand{\lap} {\nabla^2}
\renewcommand{\div}{\nabla\cdot}

\newcommand{\csch}{\text{csch}}
\newcommand{\sech}{\text{sech}}


\newcommand{\hot}{\text{h.o.t.}}

\newcommand{\ssp}{\left.\qquad\right.}

\newcommand{\var}{\text{var}}
\newcommand{\cov}{\text{cov}}


\begin{document}

\section{Blow-up of solution}
[From \cite{ShearerLevy_15}, Exercise 3.8] 

\subsection{}
Use the method of characteristics to solve the initial value problem: 
\begin{align}
    u_t+tu_x = u^2,\quad -\infty<x<\infty,\; 0<t<1
\end{align}
with initial condition 
\begin{align}
    u(x,0) = \frac{1}{1+x^2}.
\end{align}

\textit{Solution}: We have the ODE equations
\begin{align}
    &\frac{\de t}{\de \tau} = 1\\
    &\frac{\de x}{\de \tau} = t\\
    &\frac{\de u}{\de \tau} = u^2.
\end{align}
We have the solution
\begin{align}
    &t = \tau+t_0\\
    &x = \frac{\tau^2}{2}+C\label{eq:xtC}\\
    &u = \frac{-1}{\tau+D}.
\end{align}
Since the boundary data is at $t=0$, we pick $t_0=0$ for convenience. This give us $\tau = t$. We match the boundary condition
\begin{align}
    &u(x_0,0) = \frac{1}{1+x_0^2}\\
    \thus &u = \frac{-1}{t-1-x_0^2}
\end{align}
To eliminate $x_0$ we find $x_0=x(\tau=0)$ as a function of $(x,t)$. Using \eqref{eq:xtC} we have
\begin{align}
    x_0 = C = x-\frac{t^2}{2}.
\end{align}
Together we have the solution
\begin{align}
    u(x,t) = \frac{1}{1-t+(x-\frac{t^2}{2})^2}
\end{align}

\subsection{}
Show that the solution blows up as $t\to 1^-$:
\begin{align}
    \lim_{t\to 1^-} \max_x u(x,t) = \infty.
\end{align}

Remark: for a similar problem, see \cite[Exercise 2.2.11]{Olver_14}.

\textit{Solution}: Take the limit we have
\begin{align}
    \lim_{t\to 1^-} \max_x u(x,t) = \frac{1}{(x-1/2)^2}
\end{align}
and it will blow up at $x=1/2$.

    
\vfill
\printbibliography


\end{document}