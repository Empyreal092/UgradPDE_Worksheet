\documentclass[11pt,letterpaper]{article}
\usepackage[utf8]{inputenc}
\usepackage[left=1in,right=1in,top=1in,bottom=1in]{geometry}
\usepackage{amsfonts,amsmath}
\usepackage{graphicx,float}
% -----------------------------------
\usepackage{hyperref}
\hypersetup{%
  colorlinks=true,
  linkcolor=blue,
  citecolor=blue,
  urlcolor=blue,
  linkbordercolor={0 0 1}
}
% -----------------------------------
\usepackage[authordate,backend=biber]{biblatex-chicago}
\addbibresource{citation.bib}
% -----------------------------------
\usepackage{fancyhdr}
\newcommand\course{MATH-UA.0263\\Partial Differential Equations}
\newcommand\hwnumber{1}                  % <-- homework number
\newcommand\NetIDa{Ryan Sh\`iji\'e D\`u} 
\newcommand\NetIDb{February 3rd, 2022}
\pagestyle{fancyplain}
\headheight 35pt
\lhead{\NetIDa\\\NetIDb}
\chead{\textbf{\Large Worksheet \hwnumber}}
\rhead{\course}
\lfoot{}
\cfoot{}
\rfoot{\small\thepage}
\headsep 1.5em
% -----------------------------------
\usepackage{titlesec}
\renewcommand\thesubsection{(\arabic{section}.\alph{subsection})}
\titleformat{\subsection}[runin]
        {\normalfont\bfseries}
        {\thesubsection}% the label and number
        {0.5em}% space between label/number and subsection title
        {}% formatting commands applied just to subsection title
        []% punctuation or other commands following subsection title
% -----------------------------------
\setlength{\parindent}{0.0in}
\setlength{\parskip}{0.1in}
% -----------------------------------
\newcommand{\de}{\mathrm{d}}
\newcommand{\DD}{\mathrm{D}}
\newcommand{\pe}{\partial}
\newcommand{\mcal}{\mathcal}
%\newcommand{\pdx}{\left|\frac{\partial}{\partial_x}\right|}

\newcommand{\dsp}{\displaystyle}

\newcommand{\norm}[1]{\left\Vert #1 \right\Vert}
%\newcommand{\mean}[1]{\left\langle #1 \right\rangle}
\newcommand{\mean}[1]{\overline{#1}}
\newcommand{\inner}[2]{\left\langle #1,#2\right\rangle}

\newcommand{\ve}[1]{\boldsymbol{#1}}

\newcommand{\thus}{\Rightarrow \quad }
\newcommand{\fff}{\iff\quad}
\newcommand{\qdt}[1]{\quad \mbox{#1} \quad}

\renewcommand{\Re}{\mathrm{Re}}
\renewcommand{\Im}{\mathrm{Im}}
\newcommand{\E}{\mathbb{E}}
\newcommand{\lap} {\nabla^2}
\renewcommand{\div}{\nabla\cdot}

\newcommand{\csch}{\text{csch}}
\newcommand{\sech}{\text{sech}}


\newcommand{\hot}{\text{h.o.t.}}

\newcommand{\ssp}{\left.\qquad\right.}

\newcommand{\var}{\text{var}}
\newcommand{\cov}{\text{cov}}


\begin{document}

\section{Domain of dependence}
[From \cite{Olver_14}, Exercise 2.2.12] A sensor situated at position $x=1$ monitors the concentration of a pollutant $u(t,1)$ as a function of $t$ for $t\geq 0$. Assuming that the pollutant is transported with wave speed $c=3$, at what locations $x$ can you determine the initial concentration $u(0,x)$? 

Remark: this is a first example of an inverse problem. To explain a sub-class of inverse problems: ``forward'' problem is the evolution of the PDE from the initial condition, and inverse problem tries to infer information about the initial condition from observations of the solution at a later time (and a specific location). Things get significantly more difficult when diffusion, modeled by the heat equation, is in the dynamics. Inverse problem is a big field with active research. We will come back to explore more of it later on.

\section{Initial and boundary conditions}
[From \cite{Olver_14}, Exercise 2.2.14] Let $c>0$. Consider the uniform transport equation
\begin{align}
    u_t+cu_x = 0
\end{align}
restricted to the quarter-place $Q = \{x>0, t>0\}$ and subject to initial conditions
\begin{align}
    u(0,x) = f(x) \qdt{for} x\geq 0
\end{align}
along with the boundary condition
\begin{align}
    u(t,0) = g(t) \qdt{for} t\geq 0.
\end{align}

\subsection{}
For which initial and boundary conditions does a classical solution to this initial-boundary value problem exists? Write down a formula for the solution.

\subsection{}
On which regions are the effects of the initial conditions felt? What about the boundary conditions? Is there any interaction between the two?

\section{Blow-up of solution}
[From \cite{ShearerLevy_15}, Exercise 3.8] 

\subsection{}
Use the method of characteristics to solve the initial value problem: 
\begin{align}
    u_t+tu_x = u^2,\quad -\infty<x<\infty,\; 0<t<1
\end{align}
with initial condition 
\begin{align}
    u(x,0) = \frac{1}{1+x^2}.
\end{align}

\subsection{}
Show that the solution blows up as $t\to 1^-$:
\begin{align}
    \lim_{t\to 1^-} \max_x u(x,t) = \infty.
\end{align}

Remark: for a similar problem, see \cite[Exercise 2.2.11]{Olver_14}.

% \section{Transport in higher dimensions}
% [Adapted from \cite{Olver_14}, Exercise 2.2.31] We will consider transport equation in 2D. In vector form, the transport equation is
% \begin{align}
%     \frac{D\rho}{Dt}:= \frac{\pe \rho}{\pe t} + \ve u\cdot\nabla \rho = 0\label{eq:DqDt}
% \end{align}
% where we define the material derivative $D/Dt$. In 2D, the velocity vector is
% \begin{align}
%     \ve u(x,y,t) = \begin{pmatrix}
%     u(x,y,t)\\v(x,y,t)
%     \end{pmatrix}.
% \end{align}
% We also write out the dot product in \eqref{eq:DqDt}:
% \begin{align}
%     \ve u\cdot\nabla \rho = u\rho_x+v\rho_y. 
% \end{align}
% All together we have the transport equation
% \begin{align}
%     &\frac{D\rho}{Dt} = 0\\
%     \fff & \rho_t + u(x,y,t)\rho_x+v(x,y,t)\rho_y = 0.
% \end{align}

% \subsection{}
% Define a characteristic curve, and prove that the solution is constant along it.

% \subsection{}
% Apply the method of characteristics to solve the initial value problem
% \begin{align}
%     \rho_t+y\rho_x-x\rho_y = 0, \quad u(0,x,y) = e^{-(x-1)^2-(y-1)^2}.
% \end{align}

% \subsection{}
% Describe the behavior of your solution.

\section{Symmetries of the wave equation}
[From \cite{ShearerLevy_15}, Exercise 4.3] Show that if $u(x,t)\in C^3$ is a solution of the wave equation
\begin{align}
    u_{tt} = c^2u_{xx},
\end{align}
then so are the following functions:

\subsection{}
For any $y\in\mathbb{R}$, the function $u(x-y,t)$

\subsection{}
Both $u_x$ and $u_t$.

\subsection{}
For any $a\in\mathbb{R}$, the function $u(ax,at)$. 

\section{Challenging wave equation problem}
[From Spring 2019 of Applied Differential Equations qualifying exam at UCLA, Problem 1\footnote{\url{https://ww3.math.ucla.edu/wp-content/uploads/2021/09/ade-19S.pdf}}]  Let $u(x,t)$ solve the initial value problem
\begin{align}
    \begin{cases}
        u_{tt}+u_{xt}-2u_{xx} = 0,\quad x\in\mathbb{R}, t>0,\\
        u(x,0) = g(x),\\
        u_t(x,0) = h(x).
    \end{cases}
\end{align}

\subsection{}
Derive a formula for $u$ in terms of $g$ and $h$, when $g$ and $h$ are $C^2$. 

Hint: Consider how to simplify the equation into something more obviously like the wave equation
by making a change of coordinate system: $(x,t)\to (\zeta,t)$ where $\zeta = x-vt$ for $v$ appropriately determined. 

\subsection{}
Next consider the boundary value problem 
\begin{align}
    \begin{cases}
        u_{tt}+u_{xt}-2u_{xx} = 0,\quad x\in[0,1], t>0,\\
        u(0,t) = u(1,t) = 0.
    \end{cases}
\end{align}
Show that a smooth solution $u$ to above problem must be zero if $u(x,0)=u_t(x,0)=0$. 

Hint: use an energy argument. Try the energy for the wave equation. 

    
\vfill
\printbibliography


\end{document}