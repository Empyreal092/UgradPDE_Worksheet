\documentclass[11pt,letterpaper]{article}
\usepackage[utf8]{inputenc}
\usepackage[left=1in,right=1in,top=1in,bottom=1in]{geometry}
\usepackage{amsfonts,amsmath}
\usepackage{graphicx,float}
% -----------------------------------
\usepackage{hyperref}
\hypersetup{%
  colorlinks=true,
  linkcolor=blue,
  citecolor=blue,
  urlcolor=blue,
  linkbordercolor={0 0 1}
}
% -----------------------------------
\usepackage[authordate,backend=biber]{biblatex-chicago}
\addbibresource{citation.bib}
% -----------------------------------
\usepackage{fancyhdr}
\newcommand\course{MATH-UA.0263\\Partial Differential Equations}
\newcommand\hwnumber{3}                  % <-- homework number
\newcommand\NetIDa{Ryan Sh\`iji\'e D\`u} 
\newcommand\NetIDb{February 17th, 2023}
\pagestyle{fancyplain}
\headheight 35pt
\lhead{\NetIDa\\\NetIDb}
\chead{\textbf{\Large Worksheet \hwnumber}}
\rhead{\course}
\lfoot{}
\cfoot{}
\rfoot{\small\thepage}
\headsep 1.5em
% -----------------------------------
\usepackage{titlesec}
\renewcommand\thesubsection{(\arabic{section}.\alph{subsection})}
\titleformat{\subsection}[runin]
        {\normalfont\bfseries}
        {\thesubsection}% the label and number
        {0.5em}% space between label/number and subsection title
        {}% formatting commands applied just to subsection title
        []% punctuation or other commands following subsection title
% -----------------------------------
\setlength{\parindent}{0.0in}
\setlength{\parskip}{0.1in}
% -----------------------------------
\newcommand{\de}{\mathrm{d}}
\newcommand{\DD}{\mathrm{D}}
\newcommand{\pe}{\partial}
\newcommand{\mcal}{\mathcal}
%\newcommand{\pdx}{\left|\frac{\partial}{\partial_x}\right|}

\newcommand{\dsp}{\displaystyle}

\newcommand{\norm}[1]{\left\Vert #1 \right\Vert}
%\newcommand{\mean}[1]{\left\langle #1 \right\rangle}
\newcommand{\mean}[1]{\overline{#1}}
\newcommand{\inner}[2]{\left\langle #1,#2\right\rangle}

\newcommand{\ve}[1]{\boldsymbol{#1}}

\newcommand{\thus}{\Rightarrow \quad }
\newcommand{\fff}{\iff\quad}
\newcommand{\qdt}[1]{\quad \mbox{#1} \quad}

\renewcommand{\Re}{\mathrm{Re}}
\renewcommand{\Im}{\mathrm{Im}}
\newcommand{\E}{\mathbb{E}}
\newcommand{\lap} {\nabla^2}
\renewcommand{\div}{\nabla\cdot}

\newcommand{\csch}{\text{csch}}
\newcommand{\sech}{\text{sech}}


\newcommand{\hot}{\text{h.o.t.}}

\newcommand{\ssp}{\left.\qquad\right.}

\newcommand{\var}{\text{var}}
\newcommand{\cov}{\text{cov}}


\begin{document}

\section{First time of shock}
[See \cite{ShearerLevy_15}, \S 3.4.1] This is an alternative way to derive the first time of shock for Burgers' equation. We have the Burgers' equation:
\begin{align}
    & u_t + uu_x = 0\label{eq:burgers}\\
    & u(x,0) = g(x).\nonumber
\end{align}

\subsection{}
Take the $x$-derivative of the Burgers' equation and derive an equation for the evolution of $u_x$. We name $v:= u_x$.

\subsection{}
Transform the above PDE into an ODE with a change of variable. Hint: try the Galilean transformation again.

\subsection{}
Solve this ODE and recover the first time of shock result you learned in lecture.

\section{Mapping to Burgers' equation}
We have a 1D conservation law
\begin{align}
    & u_t + F(u)_x = 0.\label{eq:cons_Fu}\\
    & u(x,0) = g(x).\nonumber
\end{align}
The rigorous theory of 1D conservation laws is less complicated if the flux function $F(u)$ is uniformly convex. That is, when we have
\begin{align}
    F''\geq\theta>0
\end{align}
for some constant $\theta$. Here, we will show that if $F$ is uniformly convex (and smooth), we can map the more general problem \eqref{eq:cons_Fu} to Burgers' equation \textit{assuming that we have strong solution}.

\subsection{}
Assume $v=G(u)$, derive the formula for $v_t$ and $(v^2/2)_x$.

\subsection{}
Try to match with \eqref{eq:cons_Fu}. What does $G$ has to be? 

Note: this $G$ plays an important role in the Lax-Oleinik formula, a famous result for conservation laws. For more, see \cite[\S 3.4.2]{Evans_10}.

\newpage
\section{Traveling backwards in traffic law}
We have the Lighthill-Whitham-Richards model for traffic flow
\begin{align}
    \rho_t + F(\rho)_x = 0.
\end{align}
In particular we use the Greenshields model for the flux:
\begin{align}
    F(\rho) = \rho(1-\rho).
\end{align}

\subsection{}
What is the speed of characteristics for the Greenshields model?

\subsection{}
Notice that when the traffic density is large ($\rho>1/2$), the characteristics actually travel backwards (against the direction of flow of cars). Can you explain the backward movement of characteristics physically?

\section{Integral solution to 1D conservation law}
Note: this is above and beyond what you need for this undergraduate level class. We would talk about it only if we have time. But learning the rigorous theory is useful if you need to extend what we learned in this class to a more general context. Test functions will show up later in class. Lastly, thinking about this is a fun intellectual exercise. :)

[See \cite{Evans_10}, \S 3.4.1] At a shock, classic notion of solution to a PDE breaks down since we cannot take derivatives of the solution. We must devise some way to interpret a less regular function $u$ as somehow ``solving'' the PDE. The idea of an integral (weak) solution is to transfer the derivatives to a test function $v$ via integration by parts. 

More precisely, assume
\begin{align}
    v: \mathbb{R}\times[0,\infty)\to\mathbb{R} \text{ is smooth ($C^\infty$), with compact support}.\label{eq:test_func}
\end{align}
We say that $u\in L^\infty(\mathbb{R}\times(0,\infty))$ is an integral solution of \eqref{eq:cons_Fu}, if
\begin{align}
    \int^\infty_0\int^\infty_{-\infty} uv_t+F(u)v_x\;\de x\de t+\int^\infty_{-\infty} gv\;\de x|_{t=0} = 0
\end{align}
for all test function $v$ satisfying \eqref{eq:test_func}.

Note: For this to work such $v$ needs to exist. One example of such $v$ is the ``bump function''. Their existence might be surprising for a first-timer, but they are essential tools in analysis of PDE so get used to them.

\subsection{}
Assume $u$ is smooth, use integration by part to show that the above condition for the integral solution is equivalent to the requirement for $u$ to be a classic solution.

Note: here we use a theorem that states: for a continuous function $f$, if $f(x_0)>0$ for some $x_0$, there exists a small ball around $x_0$ such that for all $y\in B(x_0,\epsilon)$, $f(y)>0$. Review your analysis notes, or look this up.

\subsection{}
Now assume that $u$ has a jump. Perform integration by parts again to obtain the Rankine-Hugoniot condition.

To do this, assume $v$ is zero at $t=0$ to make the initial condition term zero. More importantly, the boundary terms from integration by parts are not zero at the jump. They will give you the speed of the shock.

\subsection{}
That was a lot of work. Is this enough? Unfortunately no. The integral solutions are not unique, especially for rarefaction wave. We also need the entropy condition. You will learn this next week.

    
\vfill
\printbibliography


\end{document}